% برای چاپ پایان‌نامه به صورت دو رو خط فوق را کامنت و خط زیر را فعال کنید همچنین تغییرات لازم برای هدر‌ها را نیز انجام دهید که در ادامه به آن اشاره شده است
%  \documentclass[a4paper,11pt,twoside,openany]{thesis}
\usepackage{amsthm,amssymb,amsmath,textcomp}% فونت‌ها، نمادها و محیط‌های ams
\usepackage{setspace,xargs}
\usepackage{array}%آرایه‌های ریاضی
\usepackage{verbatim}%می‌توان محیط های جدید را با این بسته تعریف نمود
\usepackage{verbatimbox}
\usepackage{indentfirst} %جهت ایجاد تورفتگی در اول پاراگراف
\usepackage{xfrac}
% بسته زیر برای جداول است
\usepackage{tabulary}
\usepackage{colortbl}
\usepackage{framed} 
% بسته زیر برای تنظیمات هدر صفحات است
\usepackage{fancyhdr}
% تنظیم Heading
\usepackage{longtable}
% پکیج برای جداول طولانی
\usepackage{enumitem}
% محیط شمارنده
\usepackage{multicol}
\setlength{\columnsep}{1cm}
% پکیج برای چند ستونی سازی 
%===============================================footnote=====================================
%بسته و تنظیمات زیر زیرنویس هر صفحه را از یک شروع می کند.
\usepackage{zref-perpage}
\zmakeperpage{footnote}
\usepackage{remreset}
\makeatletter
\@removefromreset{footnote}{chapter}
\makeatother
%======================================================================================
\usepackage{tikz,tikz-cd}% برای رسم اشکال و یا نمودارها استفاده می شود. این بسته یکی از مهمترین بسته‌های لاتک است
\usetikzlibrary{shapes.geometric, arrows,patterns}

\usepackage [pagebackref=true, colorlinks, linkcolor=blue, citecolor=magenta, urlcolor=cyan] {hyperref}
% چنانچه قصد پرینت گرفتن نوشته خود را دارید، خط بالا را غیرفعال و  از دستور زیر استفاده کنید. در ضمن pagebackref برای نشان دادن شماره صفحه ارجاعات مراجع در بخش bibliography است.
% \usepackage [pagebackref=false, colorlinks, linkcolor=black, citecolor=black, urlcolor=black] {hyperref}

\usepackage{afterpage}
\usepackage{bookmark}%برای فعال شدن لینک‌ها از این بسته استفاده می شود
% پکیج زیر رنگ و گرافیک و تعریف پوشه عکس‌ها
\usepackage{graphicx,xcolor}

\graphicspath{{./images/}}
\newcommand\figwidth{0.4}
% پکیج زیر برای اضافه کردن کدهای برنامه‌هاست
\usepackage[procnames]{listings}
 \usepackage{lscape}% چنانچه بخواهید صفحه ای را به صورت لندسکیپ درآورید این بسته کمک می کند
\usepackage [a4paper, bindingoffset=-.5cm, footskip=1cm, headheight = 16pt, top=3cm, bottom=2.5cm,  right=3cm,  left=3cm ,] {geometry}% ابعاد صفحه و حاشیه‌ها
% تنظیم ارجاعات
\usepackage[numbers]{natbib}%این بسته برای اضافه نمودن دستورات مرجع زنی مختلف است
% بسته زیر فهرست های  و مراجع را به فهرست مطالب اضافه می کند
\usepackage[nottoc]{tocbibind}
% دو بسته زیر امکان caption را برای عکس‌ها فراهم می نماید
\usepackage[margin=10pt,font=small,labelfont=bf,labelsep=endash]{caption} 
\usepackage[margin=10pt,font=footnotesize,labelfont=bf,labelsep=endash]{subcaption}
\usepackage[xindy,acronym,toc]{glossaries}% اضافه کردن مراجع و نمایه به فهرست مطالب
%بسته زیر امکان ارجاع دهی الکترونیک به مقالات را ایجاد می کند .البته باید استایل مورد استفاده در بخش مراجع دارای تابع doi باشد.استایل های iut-fa و plainnat-faاین آپشن را دارا هستند.
\usepackage{doi}
% خط زیر امکان نوشتن کنار عکس را می دهد
\usepackage{sidecap}
\sidecaptionvpos{figure}{t}
%خط زیر برای خوانش فونت‌ها در ویندوز است.
\usepackage[OT1,EU1]{fontenc}
%خط زیر مراجع را از اولین فصل شماره گذاری می کند و در لیست تصاویر نشان نمی‌دهد.
\usepackage{notoccite}
% در مورد تقدم و تاخر وارد کردن بسته ها تنها باید به چند نکته دقت کرد:
% الف) بسته xepersian حتما حتما باید آخرین بسته ای باشد که فراخوانی می شود
% ب) بسته hyperref جزو آخرین بسته هایی باید باشد که فراخوانی می شود.
% ج) بسته glossaries حتما باید بعد از hyperref فراخوانی شود. 
%  اگر از بسته float استفاده نمی کنید caption جداول مانند تصاویر بسته به اینکه بالا نوشته شده باشند یا پایین تغییر مکان می دهند. چنانچه نیازمندید تا از بسته‌های float که در زیر آمده است استفاده کنید زیرنویس جداول همه در پایین نوشته می شود. برای اینکه زیرنویس‌ها بعد از فعالسازی بسته های float بالا یا پایین جداول نوشته شود حسب انتخاب باید قبل از table مکان را با یکی از دستورات زیر ست نمایید توجه کنید که بعد از دستورات زیر تمامی زیرنویس ها از آن به بعد مطابق با آخرین دستور اعمالی تنظیم می شوند. برای نمونه به جداول فصل چهارم نگاه کنید.
% \usepackage{floatrow}
% \usepackage{morefloats}
% \floatsetup[table]{capposition=bottom}
% \floatsetup[table]{capposition=top}


%%========= Added by Ali =========%%
%\usepackage{subfigure}
%\usepackage{titling}
\newenvironment{rcases}
{\left.\begin{aligned}}
	{\end{aligned}\right\rbrace}

%\usepackage{subfigure}

%%================================%%



%برای نشان دادن رد ماتریس از این عبارت تعریف شده است.می‌توانید عبارات خود را تعریف کنید.
% خط زیر اپراتور تریس را ست می نماید
\DeclareMathOperator{\Tr}{Tr}

%%=========================================== XePersian
%  \usepackage{xepersian}
%اگر می‌خواهید زیرنویس ها تک ستونی شود خط فوق را  فعال کنید و دو خط زیر را غیر فعال کنید

 \usepackage[extrafootnotefeatures]{xepersian}
 \twocolumnfootnotes
\settextfont[Scale=1.1]{Yas}
%\setdigitfont[Scale=1.1]{Yas}
%اگر میخواهید اعداد در فرمولها لاتین باشد خط بالا را کامنت و خط پایین را فعال کنید
\DefaultMathsDigits
\defpersianfont\nastaliq[Scale=2]{IranNastaliq}
\defpersianfont\nastaliqsmal[Scale=1]{IranNastaliq}
\defpersianfont\titr[Scale=1]{XB Titre}
\defpersianfont\traffic[Scale=1]{XM Traffic}
\defpersianfont\nil[Scale=1.3]{XB Niloofar}
% \deflatinfont\urwchl[Scale=1]{Chancery}
% ٫=========================================================
\newcommand\namad[2]{#1\dotfill\lr{#2}\\}
% برای فاصله گذاری استاندارد بین خطوط و دستورات با چند آرگومان اختیاری
%=========================================================================================
% %این خطوط اعداد پانویس‌ها را درست می کند
% \makeatletter
% \footmarkstyle{\textsuperscript{\if@RTL\else\latinfont\fi#1}}
% \makeatother

\makeatletter
\def\@makeLTRfnmark{\hbox{\@textsuperscript{\latinfont\@thefnmark}}}
\renewcommand\@makefntext[1]{%
    \parindent 1em%
    \noindent
    \hb@xt@1.8em{\hss\if@RTL\@makefnmark\else\@makeLTRfnmark\fi}#1}
\makeatother
% %============================================= Counters
% \def\thesection{\arabic{section}-\thechapter}
% \def\thesubsection{\arabic{subsection}-\thesection}
% \def\theequation{\arabic{equation}-\thechapter}
% \def\thetheorem{\arabic{theorem}-\thesection}
% \def\thefigure{\arabic{figure}-\thechapter}
% \def\thetable{\arabic{table}-\thechapter}
% \def\imagetop#1{\vtop{\null\hbox{#1}}}
%\numberwithin{equation}{section}

%
% %خطوط زیر برای تغییر در شکل خط بالای سر پانویسهاست
% \renewcommand{\footnoterule}{%
% 
%   \kern -3pt
%   \hrule width 0.65\textwidth height 0.85pt
%   \kern 2pt
% }
%%%%%%%%%%%%%%%%%%%%%%%%%%%%%%%%%%%%%%%%%%%%%%%%%%%%%%%%%%%%%%%%%%%%%%%%%%%%%%%%%
%

%%%%%%%%%%%%%%%%%%%
\makeatletter
 \def\abj@num@i#1{%
   \ifcase#1\or الف \or ب\or ج\or د%
            \or ه‍\or و\or ز\or ح\or ط\fi
   \ifnum#1=\z@\abjad@zero\fi}   
 \def\@harfi#1{\ifcase#1\or الف\or ب\or پ\or ت\or ث\or
 ج\or چ\or ح\or خ\or د\or ذ\or ر\or ز\or ژ\or س\or ش\or ص\or ض\or ط\or ظ\or ع\or غ\or
 ف\or ق\or ک\or گ\or ل\or م\or ن\or و\or ه\or ی\else\@ctrerr\fi}
 \def\@glsgetgrouptitle#1{\ifcase#1\or الف \or ب\or پ\or ت\or ث\or
 ج\or چ\or ح\or خ\or د\or ذ\or ر\or ز\or ژ\or س\or ش\or ص\or ض\or ط\or ظ\or ع\or غ\or
 ف\or ق\or ک\or گ\or ل\or م\or ن\or و\or ه\or ی\else\@ctrerr\fi}
\makeatother
\makeatletter
\bidi@patchcmd{\@Abjad}{آ}{الف}
{\typeout{Succeeded in changing `آ` into `الف`}}
{\typeout{Failed in changing `آ` into `الف`}}
\makeatother
\PersianAlphs
% خطوط فوق جای آرا با الف عوض می‌کنند.اگر آ را ترجیح می دهید این خطوط را غیرفعال کنید
\makeatletter 
\def\@chapter[#1]#2{\ifnum \c@secnumdepth >\m@ne
                         \refstepcounter{chapter}%
                         \typeout{\@chapapp\space\thechapter.}%
                         \addcontentsline{toc}{chapter}%
                                   {\@chapapp~\protect\numberline{\thechapter}#1}%
                    \else
                      \addcontentsline{toc}{chapter}{#1}%
                    \fi
                    \chaptermark{#1}%
                    \addtocontents{lof}{\protect\addvspace{10\p@}}%
                    \addtocontents{lot}{\protect\addvspace{10\p@}}%
                    \if@twocolumn
                      \@topnewpage[\@makechapterhead{#2}]%
                    \else
                      \@makechapterhead{#2}%
                      \@afterheading
                    \fi}
\renewcommand*\l@section{\@dottedtocline{1}{3.5em}{3.3em}}
\renewcommand*\l@subsection{\@dottedtocline{2}{4.8em}{4.2em}} 
\makeatother
% خطوط فوق تنظیمات فواصل را در فهرست مطالب انجام می دهند اعداد در سه خط پایانی این خطوط مدیر این کارند
% \SepMark{-}
% در فهرست مطالب و ارجاعات اگر می‌خواهید به جای نقطه - بگذارید از خط بالا استفاده کنید.
\makeatletter
\bidi@patchcmd{\Hy@org@chapter}{%
\addcontentsline{toc}{chapter}%
{\protect\numberline{\thechapter}#1}%
}{%
\addcontentsline{toc}{chapter}%
{\protect\numberline{\chaptername~\tartibi{chapter}}#1}%
}{\typeout{We succeded in redefining \string\@chapter}}
{\typeout{We failed in redefining \string\@chapter}}
\bidi@patchcmd{\l@chapter}{%
\setlength\@tempdima{1.5em}%
 }{%
\setlength\@tempdima{3.em}%
}{\typeout{We succeded in redefining \string\l@chapter}}
{\typeout{We failed in redefining \string\l@chapter}}
%تنظیم فاصله پیوست ها در فهرست با خطوط بالاست
%%%%%%%%%%%%%%%%%%%%%%%%%%%%%%%%%%
%%% ============================================================================================================

%%% تنظیمات مربوط به بسته  glossaries
%%% تعریف استایل برای واژه نامه فارسی به انگلیسی، در این استایل واژه‌های فارسی در سمت راست و واژه‌های انگلیسی در سمت چپ خواهند آمد. از حالت گروه ‌بندی استفاده می‌کنیم، 
%%% یعنی واژه‌ها در گروه‌هایی به ترتیب حروف الفبا مرتب می‌شوند، مثلا:
%%% الف
%%% افتصاد ................................... Economy
%%% اشکال ........................................ Failure
%%% ش
%%% شبکه ...................................... Network

\newglossarystyle{myFaToEn}{%
	\renewenvironment{theglossary}{}{}
	\renewcommand*{\glsgroupskip}{\vskip 10mm}
	\renewcommand*{\glsgroupheading}[1]{\subsection*{\glsgetgrouptitle{##1}}}
	\renewcommand*{\glossentry}[2]{\begin{flushleft}\noindent\glsentryname{##1}{،##2}\dotfill\space\glsentrytext{##1}\end{flushleft}
	}
}

%% % تعریف استایل برای واژه نامه انگلیسی به فارسی، در این استایل واژه‌های فارسی در سمت راست و واژه‌های انگلیسی در سمت چپ خواهند آمد. از حالت گروه ‌بندی استفاده می‌کنیم، 
%% % یعنی واژه‌ها در گروه‌هایی به ترتیب حروف الفبا مرتب می‌شوند، مثلا:
%% % E
%%% Economy ............................... اقتصاد
%% % F
%% % Failure................................... اشکال
%% %N
%% % Network ................................. شبکه

\newglossarystyle{myEntoFa}{%
	%%% این دستور در حقیقت عملیات گروه‌بندی را انجام می‌دهد. بدین صورت که واژه‌ها در بخش‌های جداگانه گروه‌بندی می‌شوند، 
	%%% عنوان بخش همان نام حرفی است که هر واژه در آن گروه با آن شروع شده است. 
	\renewenvironment{theglossary}{}{}
	\renewcommand*{\glsgroupskip}{\vskip 10mm}
	\renewcommand*{\glsgroupheading}[1]{\begin{LTR} \subsection*{\glsgetgrouptitle{##1}} \end{LTR}}
	%%% در این دستور نحوه نمایش واژه‌ها می‌آید. در این جا واژه فارسی در سمت راست و واژه انگلیسی در سمت چپ قرار داده شده است، و بین آن با نقطه پر می‌شود. 
	\renewcommand*{\glossentry}[2]{\begin{flushleft}\glsentrytext{##1}\noindent\dotfill\space \lr{\glsentryname{##1}{ ,##2}}\end{flushleft}	
	}
}

%%% تعیین استایل برای فهرست اختصارات
\newglossarystyle{myAbbrlist}{%
	%%% این دستور در حقیقت عملیات گروه‌بندی را انجام می‌دهد. بدین صورت که اختصارات‌ در بخش‌های جداگانه گروه‌بندی می‌شوند، 
	%%% عنوان بخش همان نام حرفی است که هر اختصار در آن گروه با آن شروع شده است. 
	\renewenvironment{theglossary}{}{}
	\renewcommand*{\glsgroupskip}{\vskip 10mm}
	\renewcommand*{\glsgroupheading}[1]{\begin{LTR} \subsection*{\glsgetgrouptitle{##1}} \end{LTR}}
	%%% در این دستور نحوه نمایش اختصارات می‌آید. در این جا حالت کوچک اختصار در سمت چپ و حالت بزرگ در سمت راست قرار داده شده است، و بین آن با نقطه پر می‌شود. 
	\renewcommand*{\glossentry}[2]{\noindent\glsentrytext{##1}\lr{##2,}\dotfill\space \Glsentrylong{##1}
		
	}
	%%% تغییر نام محیط abbreviation به فهرست اختصارات
	\renewcommand*{\acronymname}{\rl{فهرست اختصارات}}
}

%%% برای اجرا xindy بر روی فایل .tex و تولید واژه‌نامه‌ها و فهرست اختصارات و فهرست نمادها یکسری  فایل تعریف شده است.‌ Latex داده های مربوط به واژه نامه و .. را در این 
%%%  فایل‌ها نگهداری می‌کند. مهم‌ترین option‌ این قسمت این است که 
%%% عنوان واژه‌نامه‌ها و یا فهرست اختصارات و یا فهرست نمادها را می‌توانید در این‌جا مشخص کنید. 
%%% در این جا عباراتی مثل glg، gls، glo و ... پسوند فایل‌هایی است که برای xindy بکار می‌روند. 
\newglossary[glg]{english}{gls}{glo}{واژه‌نامه انگلیسی به فارسی}
\newglossary[blg]{persian}{bls}{blo}{واژه‌نامه فارسی به انگلیسی}
\makeglossaries
\glsdisablehyper
%%% تعاریف مربوط به تولید واژه نامه و فهرست اختصارات و فهرست نمادها
%%%  در این فایل یکسری دستورات عمومی برای وارد کردن واژه‌نامه آمده است.
%%%  به دلیل این‌که قرار است این دستورات پایه‌ای را بازنویسی کنیم در این‌جا تعریف می‌کنیم. 
\let\oldgls\gls
\let\oldglspl\glspl

\makeatletter

\renewrobustcmd*{\gls}{\@ifstar\@msgls\@mgls}
\newcommand*{\@mgls}[1] {\ifthenelse{\equal{\glsentrytype{#1}}{english}}{\oldgls{#1}\glsuseri{f-#1}}{\lr{\oldgls{#1}}}}
\newcommand*{\@msgls}[1]{\ifthenelse{\equal{\glsentrytype{#1}}{english}}{\glstext{#1}\glsuseri{f-#1}}{\lr{\glsentryname{#1}}}}

\renewrobustcmd*{\glspl}{\@ifstar\@msglspl\@mglspl}
\newcommand*{\@mglspl}[1] {\ifthenelse{\equal{\glsentrytype{#1}}{english}}{\oldglspl{#1}\glsuseri{f-#1}}{\oldglspl{#1}}}
\newcommand*{\@msglspl}[1]{\ifthenelse{\equal{\glsentrytype{#1}}{english}}{\glsplural{#1}\glsuseri{f-#1}}{\glsentryplural{#1}}}

\makeatother

\newcommand{\newword}[4]{
	\newglossaryentry{#1}     {type={english},name={\lr{#2}},plural={#4},text={#3},description={}}
	\newglossaryentry{f-#1} {type={persian},name={#3},text={\lr{#2}},description={}}
}

%%% بر طبق این دستور، در اولین باری که واژه مورد نظر از واژه‌نامه وارد شود، پاورقی زده می‌شود. 
\defglsentryfmt[english]{\glsgenentryfmt\ifglsused{\glslabel}{}{\LTRfootnote{\glsentryname{\glslabel}}}}

%%% بر طبق این دستور، در اولین باری که واژه مورد نظر از فهرست اختصارات وارد شود، پاورقی زده می‌شود. 
\defglsentryfmt[acronym]{\glsentryname{\glslabel}\ifglsused{\glslabel}{}{\LTRfootnote{\glsentrydesc{\glslabel}}}}


%%%%%% ================================================ دستور برای قرار دادن فهرست اختصارات 
\newcommand{\printabbreviation}{
	\cleardoublepage
	\phantomsection
	\baselineskip=.75cm
	%% با این دستور عنوان فهرست اختصارات به فهرست مطالب اضافه می‌شود. 
%	\addcontentsline{toc}{chapter}{فهرست اختصارات}
	\setglossarystyle{myAbbrlist}
	\begin{LTR}
		\Oldprintglossary[type=acronym]	
	\end{LTR}
	\clearpage
}%

\newcommand{\printacronyms}{\printabbreviation}
%%% در این جا محیط هر دو واژه نامه را باز تعریف کرده ایم، تا اولا مشکل قرار دادن صفحه اضافی را حل کنیم، ثانیا عنوان واژه نامه ها را با دستور addcontentlist وارد فهرست مطالب کرده ایم.
\let\Oldprintglossary\printglossary
\renewcommand{\printglossary}{
	\let\appendix\relax
	%% تنظیم کننده فاصله بین خطوط در این قسمت
	\clearpage
	\phantomsection
	%% این دستور موجب این می‌شود که واژه‌نامه‌ها در  حالت دو ستونی نوشته شود. 
	\twocolumn{}
	%% با این دستور عنوان واژه‌نامه به فهرست مطالب اضافه می‌شود. 
% 	\addcontentsline{toc}{chapter}{واژه نامه انگلیسی به فارسی}
	\setglossarystyle{myEntoFa}
	\Oldprintglossary[type=english]
	
	\clearpage
	\phantomsection
	%% با این دستور عنوان واژه‌نامه به فهرست مطالب اضافه می‌شود. 
% 	\addcontentsline{toc}{chapter}{واژه نامه فارسی به انگلیسی}
	\setglossarystyle{myFaToEn}
	\Oldprintglossary[type=persian]
	\onecolumn{}
}%
%%%%%% ============================================================================================================
%%%%%% 

%
%%============================================ Titles
\renewcommand{\abstractname}{\Large چکیده}
\renewcommand{\listfigurename}{فهرست تصاویر}
%\renewcommand{\latinabstract}{}
\renewcommand{\proofname}{\textbf{برهان}}
\renewcommand{\qedsymbol}{$\blacksquare$}
\renewcommand{\bibname}{مراجع}

% \newcommand*{\doi}[1]{doi:\href*{http://dx.doi.org/#1}{#1}}
% for figures: caption label is italic, the caption text is bold / italic
%\captionsetup[figure]{labelfont={bf,it},textfont={normalfont,it}}
% for subfigures: caption label is bold, the caption text normal.
% justification is raggedright (i.e. left aligned)
% singlelinecheck=off means that the justification setting is used even when the caption is only a single line long. 
% if singlelinecheck=on, then caption is always centered when the caption is only one line.
%\captionsetup[subfigure]{labelfont=normalfont,textfont=bf,singlelinecheck=off,justification=raggedright}
%\captionsetup{textfont=rm,justification=centering,labelsep=newline}
%%======================================== Environments
\newcounter{theorem}[section]
\newcommand{\environ}[2]{\vspace{7pt} \refstepcounter{theorem}\par\noindent  \textbf{\hboxR{#1}\space\thetheorem} \textbf{\space\hboxR{#2}} \\[5pt]}
\newcommand{\closeenviron}{\par\vspace{3pt}}
\newenvironment{thm}[1][]{\environ{قضیه}{#1}\it}{\closeenviron}
\newenvironment{lem}[1][]{\environ{لم}{#1}\it}{\closeenviron}
\newenvironment{prop}[1][]{\environ{گزاره}{#1}\it}{\closeenviron}
\newenvironment{cor}[1][]{\environ{نتیجه}{#1}\it}{\closeenviron}
\newenvironment{con}[1][]{\environ{حدس}{#1}\it}{\closeenviron}
\newenvironment{dfn}[1][]{\environ{تعریف}{#1}\rm}{‌\hfill $\blacktriangle$ \closeenviron}
\newenvironment{notation}[1][]{\environ{نماد}{#1}\rm}{‌\hfill $\blacktriangledown$ \closeenviron}
\newenvironment{rem}[1][]{\environ{ملاحظه}{#1}\rm}{‌\hfill $\blacklozenge$ \closeenviron}
\newenvironment{exm}[1][]{\environ{مثال}{#1}\rm}{‌\hfill $\bigstar$ \closeenviron}
%
\makeatletter
\newenvironment{prob}[4][]{\@ifempty{#1}
{\vspace{15pt} \par\noindent
\parbox{15cm}{\hskip 7pt\underline{\bf #2}\\[4pt]
\begin{tabular}{p{40pt}l}
\textbf{نمونه:}& \parbox[t]{11.8cm}{#3}\\[7pt]
\textbf{سوال:}& \parbox[t]{11.8cm}{#4}
\end{tabular}\vspace{5pt}
}}
{\vspace{5pt} \par\noindent
\parbox{15cm}{\hskip 7pt\underline{\bf #2}\\[4pt]
\begin{tabular}{p{40pt}l}
\textbf{ثابت‌ها:}& \parbox[t]{11.5cm}{#1}\\[4pt]
\textbf{نمونه:}& \parbox[t]{11.5cm}{#3}\\[7pt]
\textbf{سوال:}& \parbox[t]{11.5cm}{#4}
\end{tabular}\vspace{5pt}
}}} {\\[5pt]}
\makeatother

%

%%======================================== Main body
%


\headheight = 20pt
\pagestyle{plain}
\fancyhf{}
% \lhead{\thepage}
% \rhead{\leftmark}
\doublespacing
\allowdisplaybreaks[1]
% اجازه برای شکستن صفحه در وسط محیط ریاضی
%\setlength{\parindent}{1cm} %دستور برای مشخص کردن فاصله ابتدای هر پاراگراف

% %%%%% ============================================================================================================
% % فرامین مربوط به تعیین رنگ و استفاده از آنها در قسمت کد های برنامه نویسی
\definecolor{keywords}{RGB}{255,0,90}
\definecolor{comments}{RGB}{0,0,205}
\definecolor{red}{RGB}{160,0,0}
\definecolor{green}{RGB}{0,150,0}
 \definecolor{Background}{rgb}{0.98,0.98,0.98}
 \definecolor{Keywords}{rgb}{0,0,1}
  \definecolor{Black}{RGB}{0,0,0}
 \definecolor{VioletRed}{RGB}{208,32,144}
 \definecolor{DarkOliveGreen}{RGB}{85,107,47}
 \definecolor{Saddle Brown}{RGB}{139,69,19}
 \definecolor{juliacomment}{RGB}{204,204,0}
 
%  تنظیمات زیر برای رنگ بندی کد بش است
  \lstdefinestyle{Mybash}{ language = Bash,
  literate = {\$\#}{{{\$\#}}}2,
  columns  = fullflexible,
 basicstyle=\ttfamily\scriptsize, 
        keywordstyle=\color{DarkOliveGreen}\bfseries,
        commentstyle=\color{comments}\bfseries,
        stringstyle=\color{green}\bfseries,
        showstringspaces=false,
        identifierstyle=\color{Black}\bfseries,
        procnamekeys={cp,sudo,chmod,fplo2wannier},
        prebreak=\raisebox{0ex}[0ex][0ex]{\ensuremath{\hookleftarrow}},
        numbers=left,
        numberstyle=\footnotesize\color{Saddle Brown},
        breaklines=true,  
        numbersep=5pt,
        captionpos=b,   
        backgroundcolor=\color{Background}\bfseries,
        tabsize=2,
        morekeywords={[2]},
        keywordstyle={[2]\color{VioletRed}\bfseries},
        morekeywords={[3],kmeshfplo,cp,bash,sudo,chmod,fplo2wannier,apt-get},
        keywordstyle={[3]\color{DarkOliveGreen}\bfseries},
        emph={self},
        emphstyle={\color{self}\bfseries},
        frame=1
	}
 \lstdefinestyle{Tex}{ language = Tex,
  literate = {\$\#}{{{\$\#}}}2,
  columns  = fullflexible,
    escapeinside={\%*}{*)},
      morekeywords={encoding,
        xs:schema,xs:element,xs:complexType,xs:sequence,xs:attribute},
 basicstyle=\ttfamily\scriptsize, 
        keywordstyle=\color{DarkOliveGreen}\bfseries,
        commentstyle=\color{comments}\bfseries,
        stringstyle=\color{green}\bfseries,
        showstringspaces=false,
        identifierstyle=\color{Black}\bfseries,
         procnamekeys={end,begin,documentclass,usepackage},
        prebreak=\raisebox{0ex}[0ex][0ex]{\ensuremath{\hookleftarrow}},
        numbers=left,
        numberstyle=\footnotesize\color{Saddle Brown},
        breaklines=true,  
        numbersep=5pt,
        captionpos=b,   
        backgroundcolor=\color{Background}\bfseries,
        tabsize=2,
        morekeywords={[2]amsmath,amssymb,amsthm,array,babel,biblatex,bm,booktabs,boxedminipage,caption,cancel,chemmacros,changepage,cleveref,dcolumn,enumitem,epstopdf,esint,eucal,fancyhdr,float,fontenc,gensymb,geometry,glossaries,graphicx,grffile,hyperref,indentfirst,inputenc,latexsym,listings,longtable,mathptmx,mathrsfs,mathtools,mhchem,microtype,multicol,natbib,pdfpages,rotating,setspace,showkeys,showidx,subfiles,subcaption,syntonly,textcomp,theorem,todonotes,siunitx,ulem,url,verbatim,xcolor,xypic},
        keywordstyle={[2]\color{VioletRed}\bfseries},
%         morekeywords={[3]},
%         keywordstyle={[3]end,begin,documentclass,usepackage \color{DarkOliveGreen}\bfseries},
        emph={self},
        emphstyle={\color{self}\bfseries},
        frame=1
	}
% تنظیمات زیر برای رنگبندی کد Python‌است
 \lstdefinestyle{Mypython}{language=Python, 
 basicstyle=\ttfamily\scriptsize, 
        sensitive=true,
        keywordstyle=\color{Keywords}\bfseries,
        commentstyle=\color{comments}\bfseries,
        stringstyle=\color{green}\bfseries,
        showstringspaces=false,
        identifierstyle=\color{Black}\bfseries,
        procnamekeys={def,class},
        prebreak=\raisebox{0ex}[0ex][0ex]{\ensuremath{\hookleftarrow}},
        numbers=left,
        numberstyle=\footnotesize\color{Saddle Brown},
        breaklines=true,  
        numbersep=5pt,
        captionpos=b,   
        backgroundcolor=\color{Background}\bfseries,
        tabsize=2,
        morekeywords={[2]@invariant},
        keywordstyle={[2]\color{VioletRed}\bfseries},
        morekeywords={[3],reshape,sin,cos,exp,conjugate,shape,pi,sqrt,genfromtxt,isclose,arctan,complex,dot,arccos,array,float64,sum,multiply,divide,subtract,add,nan_to_num,arctan2,zeros},
        keywordstyle={[3]\color{DarkOliveGreen}\bfseries},
        emph={self},
        emphstyle={\color{self}\bfseries},
        frame=1
	}
% تنظیمات برای رنگ بندی کد C است
\definecolor{mGreen}{rgb}{0,0.6,0}
\definecolor{mGray}{rgb}{0.5,0.5,0.5}
\definecolor{mPurple}{rgb}{0.58,0,0.82}
\definecolor{backgroundColour}{rgb}{0.95,0.95,0.92}
\definecolor{mygreen}{RGB}{28,172,0} % color values Red, Green, Blue
\definecolor{mylilas}{RGB}{170,55,241}
\lstdefinestyle{CStyle}{
    backgroundcolor=\color{backgroundColour},   
    commentstyle=\color{mGreen},
    keywordstyle=\color{magenta},
    numberstyle=\scriptsize\color{Saddle Brown},
    stringstyle=\color{mPurple},
 basicstyle=\ttfamily\scriptsize, 
    breakatwhitespace=false,         
    breaklines=true,                 
    captionpos=b,                    
    keepspaces=true,                 
    numbers=left,                    
    numbersep=5pt,                  
    showspaces=false,                
    showstringspaces=false,
    showtabs=false,                  
    tabsize=2,
    language=C
}
\lstdefinestyle{julia}
{
  keywordsprefix=\@,
  morekeywords={,exit,whos,edit,load,is,isa,isequal,typeof,tuple,ntuple,uid,hash,finalizer,convert,promote,
    subtype,typemin,typemax,realmin,realmax,sizeof,eps,promote_type,method_exists,applicable,
    invoke,dlopen,dlsym,system,error,throw,assert,new,Inf,Nan,pi,im,begin,while,for,in,return,
    break,continue,macro,quote,let,for,function,println,while,
    if,elseif,else,try,catch,end,bitstype,ccall,do,using,module,
    import,export,importall,baremodule,immutable,local,global,const,Bool
  },
  sensitive=true,
  alsoother={$},%
   morecomment=[l]\#,%
   morecomment=[n]{\#=}{=\#},%
   morestring=[s]{"}{"},%
   morestring=[m]{'}{'},%
basicstyle=\ttfamily\scriptsize, 
        sensitive=true,
        keywordstyle=\color{Keywords}\bfseries,
        commentstyle=\color{juliacomment}\bfseries,
        stringstyle=\color{green}\bfseries,
        showstringspaces=false,
        identifierstyle=\color{Black}\bfseries,
        procnamekeys={def,class},
        prebreak=\raisebox{0ex}[0ex][0ex]{\ensuremath{\hookleftarrow}},
        numbers=left,
        numberstyle=\footnotesize\color{Saddle Brown},
        breaklines=true,  
        numbersep=5pt,
        captionpos=b,   
        backgroundcolor=\color{backgroundColour}\bfseries,
         morekeywords={[3],reshape,sin,cos,exp,conjugate,shape,pi,sqrt,genfromtxt,isclose,arctan,complex,dot,arccos,array,float64,sum,multiply,divide,subtract,add,nan_to_num,arctan2,zeros,Int,Int8,Int16,Int32,rand,randn,trace,lolog,diag,eye,linespace,grid
    Int64,Uint,Uint8,Uint16,Uint32,Uint64,Float32,Float64,Complex64,Complex128,Any,Nothing,None,type,typealias,abstract},
           keywordstyle={[3]\color{DarkOliveGreen}\bfseries},
        tabsize=2,
        emph={self},
        emphstyle={\color{self}\bfseries},
        frame=1
}
\definecolor{mygreen}{rgb}{0,0.6,0}
\definecolor{mygray}{rgb}{0.5,0.5,0.5}
\definecolor{mymauve}{rgb}{0.58,0,0.82}
\lstdefinestyle{mymatlab}{language=Matlab,%
  basicstyle=\ttfamily\scriptsize,         % size of fonts used for the code
    %basicstyle=\color{red},
    backgroundcolor=\color{backgroundColour},   
    breaklines=true,%
    morekeywords={matlab2tikz},
      captionpos=b,                    % sets the caption-position to bottom
    keywordstyle=\color{blue},%
%     morekeywords=[2]{1}, keywordstyle=[2]{\color{black}},
    identifierstyle=\color{black},%
    stringstyle=\color{mylilas},
    commentstyle=\color{mygreen},%
    showstringspaces=false,%without this there will be a symbol in the places where there is a space
    numbers=left,%
    numberstyle={\tiny \color{black}},% size of the numbers
    numbersep=9pt, % this defines how far the numbers are from the text
    emph=[1]{for,end,break},emphstyle=[1]\color{red}, %some words to emphasise
    numberstyle=\footnotesize\color{Saddle Brown}
    %emph=[2]{word1,word2}, emphstyle=[2]{style},    
}
\lstdefinestyle{myfortran}{language=[90]Fortran,
backgroundcolor=\color{backgroundColour},   
  morecomment=[l]{!\ }% Comment only with space after !
   backgroundcolor=\color{white},   % choose the background color
 basicstyle=\ttfamily\scriptsize,         % size of fonts used for the code
  breaklines=true,                 % automatic line breaking only at whitespace
  captionpos=b,                    % sets the caption-position to bottom
  commentstyle=\color{mygreen},    % comment style
  escapeinside={\%*}{*)},          % if you want to add LaTeX within your code
  keywordstyle=\color{blue},       % keyword style
  stringstyle=\color{mymauve},     % string literal style
  numbers=left,
  numberstyle=\footnotesize\color{Saddle Brown},
}
%=====================================================

\XeTeXinterchartokenstate=1
%دستوری برای اینکه کشیدگی در کلمات ایجاد شود
\abovedisplayshortskip=10pt
\belowdisplayshortskip=8pt
%دستوری برای تنظیم فاصله عمودی قبل و بعد از فرمول‌ها